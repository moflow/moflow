\section{\texttt{toil} and \texttt{get\_functions}: Lifting Binary Code to \bil}

The \texttt{toil} and \texttt{get\_functions} tools both lift binary
code to \bil.  Lifting up binary code to the IL consists of two steps:
\begin{itemize}
\item First, the binary file is opened using libbfd. BAP can read
  binaries from any bfd-supported binary, including ELF and PE
  binaries.

\item Second, the executable code in the binary is disassembled.
  \texttt{toil} currently uses a linear sweep disassembler, whereas
  \texttt{get\_functions} uses more advanced CFG recovery algorithms.

\item Each assembly instruction discovered during disassembly is then
  lifted directly to \bil.
\end{itemize}

Lifted assembly instructions have all of the side-effects explicitly
exposed.  As a result, a single typical assembly instruction will be
lifted as a sequence of \bil instructions.  For example, the {\tt add
  \$2, \%eax} instruction is lifted as:

\begin{centering}
\begin{scriptsize}
\verbatiminput{chap-tools/add2.il}
\end{scriptsize}
\end{centering}
%
The lifted \bil code explicitly detail all the side-effects of the
{\tt add} instruction, including all six flags that are updated by the
operation.

In addition to binary files, {\tt toil} can also lift an instruction
trace to the IL.  The most recent BAP trace format can be lifted using
the {\tt -trace} option.

{\tt toil} can output to several formats for easy parsing, including
protobuf, JSON, and XML.  These formats are selected via the {\tt
  -topb}, {\tt -tojson}, and {\tt -toxml} options.  Code for reading
the protobuf encoding is located in the {\tt piqi-files/protobuf}
directory.

The \texttt{get\_functions} utility identifies functions in the
designated binary and, for each function, outputs files in the current
directory containing the BIL representation and a CFG representation.
Unlike \texttt{toil}, \texttt{get\_functions} employs CFG recovery
algorithms instead of linear disassembly.  In practice, this means
that \texttt{get\_functions} can be used to recover BIL for functions
that use indirect jumps to implement switches.


%%% Local Variables: 
%%% mode: latex
%%% TeX-master: "../main"
%%% End: 
